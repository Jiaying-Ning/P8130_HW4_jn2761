\PassOptionsToPackage{unicode=true}{hyperref} % options for packages loaded elsewhere
\PassOptionsToPackage{hyphens}{url}
%
\documentclass[]{article}
\usepackage{lmodern}
\usepackage{amssymb,amsmath}
\usepackage{ifxetex,ifluatex}
\usepackage{fixltx2e} % provides \textsubscript
\ifnum 0\ifxetex 1\fi\ifluatex 1\fi=0 % if pdftex
  \usepackage[T1]{fontenc}
  \usepackage[utf8]{inputenc}
  \usepackage{textcomp} % provides euro and other symbols
\else % if luatex or xelatex
  \usepackage{unicode-math}
  \defaultfontfeatures{Ligatures=TeX,Scale=MatchLowercase}
\fi
% use upquote if available, for straight quotes in verbatim environments
\IfFileExists{upquote.sty}{\usepackage{upquote}}{}
% use microtype if available
\IfFileExists{microtype.sty}{%
\usepackage[]{microtype}
\UseMicrotypeSet[protrusion]{basicmath} % disable protrusion for tt fonts
}{}
\IfFileExists{parskip.sty}{%
\usepackage{parskip}
}{% else
\setlength{\parindent}{0pt}
\setlength{\parskip}{6pt plus 2pt minus 1pt}
}
\usepackage{hyperref}
\hypersetup{
            pdftitle={p8130\_hw4\_jn2761},
            pdfauthor={jiaying Ning},
            pdfborder={0 0 0},
            breaklinks=true}
\urlstyle{same}  % don't use monospace font for urls
\usepackage[margin=1in]{geometry}
\usepackage{color}
\usepackage{fancyvrb}
\newcommand{\VerbBar}{|}
\newcommand{\VERB}{\Verb[commandchars=\\\{\}]}
\DefineVerbatimEnvironment{Highlighting}{Verbatim}{commandchars=\\\{\}}
% Add ',fontsize=\small' for more characters per line
\usepackage{framed}
\definecolor{shadecolor}{RGB}{248,248,248}
\newenvironment{Shaded}{\begin{snugshade}}{\end{snugshade}}
\newcommand{\AlertTok}[1]{\textcolor[rgb]{0.94,0.16,0.16}{#1}}
\newcommand{\AnnotationTok}[1]{\textcolor[rgb]{0.56,0.35,0.01}{\textbf{\textit{#1}}}}
\newcommand{\AttributeTok}[1]{\textcolor[rgb]{0.77,0.63,0.00}{#1}}
\newcommand{\BaseNTok}[1]{\textcolor[rgb]{0.00,0.00,0.81}{#1}}
\newcommand{\BuiltInTok}[1]{#1}
\newcommand{\CharTok}[1]{\textcolor[rgb]{0.31,0.60,0.02}{#1}}
\newcommand{\CommentTok}[1]{\textcolor[rgb]{0.56,0.35,0.01}{\textit{#1}}}
\newcommand{\CommentVarTok}[1]{\textcolor[rgb]{0.56,0.35,0.01}{\textbf{\textit{#1}}}}
\newcommand{\ConstantTok}[1]{\textcolor[rgb]{0.00,0.00,0.00}{#1}}
\newcommand{\ControlFlowTok}[1]{\textcolor[rgb]{0.13,0.29,0.53}{\textbf{#1}}}
\newcommand{\DataTypeTok}[1]{\textcolor[rgb]{0.13,0.29,0.53}{#1}}
\newcommand{\DecValTok}[1]{\textcolor[rgb]{0.00,0.00,0.81}{#1}}
\newcommand{\DocumentationTok}[1]{\textcolor[rgb]{0.56,0.35,0.01}{\textbf{\textit{#1}}}}
\newcommand{\ErrorTok}[1]{\textcolor[rgb]{0.64,0.00,0.00}{\textbf{#1}}}
\newcommand{\ExtensionTok}[1]{#1}
\newcommand{\FloatTok}[1]{\textcolor[rgb]{0.00,0.00,0.81}{#1}}
\newcommand{\FunctionTok}[1]{\textcolor[rgb]{0.00,0.00,0.00}{#1}}
\newcommand{\ImportTok}[1]{#1}
\newcommand{\InformationTok}[1]{\textcolor[rgb]{0.56,0.35,0.01}{\textbf{\textit{#1}}}}
\newcommand{\KeywordTok}[1]{\textcolor[rgb]{0.13,0.29,0.53}{\textbf{#1}}}
\newcommand{\NormalTok}[1]{#1}
\newcommand{\OperatorTok}[1]{\textcolor[rgb]{0.81,0.36,0.00}{\textbf{#1}}}
\newcommand{\OtherTok}[1]{\textcolor[rgb]{0.56,0.35,0.01}{#1}}
\newcommand{\PreprocessorTok}[1]{\textcolor[rgb]{0.56,0.35,0.01}{\textit{#1}}}
\newcommand{\RegionMarkerTok}[1]{#1}
\newcommand{\SpecialCharTok}[1]{\textcolor[rgb]{0.00,0.00,0.00}{#1}}
\newcommand{\SpecialStringTok}[1]{\textcolor[rgb]{0.31,0.60,0.02}{#1}}
\newcommand{\StringTok}[1]{\textcolor[rgb]{0.31,0.60,0.02}{#1}}
\newcommand{\VariableTok}[1]{\textcolor[rgb]{0.00,0.00,0.00}{#1}}
\newcommand{\VerbatimStringTok}[1]{\textcolor[rgb]{0.31,0.60,0.02}{#1}}
\newcommand{\WarningTok}[1]{\textcolor[rgb]{0.56,0.35,0.01}{\textbf{\textit{#1}}}}
\usepackage{graphicx,grffile}
\makeatletter
\def\maxwidth{\ifdim\Gin@nat@width>\linewidth\linewidth\else\Gin@nat@width\fi}
\def\maxheight{\ifdim\Gin@nat@height>\textheight\textheight\else\Gin@nat@height\fi}
\makeatother
% Scale images if necessary, so that they will not overflow the page
% margins by default, and it is still possible to overwrite the defaults
% using explicit options in \includegraphics[width, height, ...]{}
\setkeys{Gin}{width=\maxwidth,height=\maxheight,keepaspectratio}
\setlength{\emergencystretch}{3em}  % prevent overfull lines
\providecommand{\tightlist}{%
  \setlength{\itemsep}{0pt}\setlength{\parskip}{0pt}}
\setcounter{secnumdepth}{0}
% Redefines (sub)paragraphs to behave more like sections
\ifx\paragraph\undefined\else
\let\oldparagraph\paragraph
\renewcommand{\paragraph}[1]{\oldparagraph{#1}\mbox{}}
\fi
\ifx\subparagraph\undefined\else
\let\oldsubparagraph\subparagraph
\renewcommand{\subparagraph}[1]{\oldsubparagraph{#1}\mbox{}}
\fi

% set default figure placement to htbp
\makeatletter
\def\fps@figure{htbp}
\makeatother


\title{p8130\_hw4\_jn2761}
\author{jiaying Ning}
\date{11/6/2020}

\begin{document}
\maketitle

\begin{Shaded}
\begin{Highlighting}[]
\KeywordTok{rm}\NormalTok{(}\DataTypeTok{list=}\KeywordTok{ls}\NormalTok{())}
\end{Highlighting}
\end{Shaded}

\begin{Shaded}
\begin{Highlighting}[]
\KeywordTok{library}\NormalTok{(readxl)}
\KeywordTok{library}\NormalTok{(tidyverse)}
\end{Highlighting}
\end{Shaded}

\begin{verbatim}
## -- Attaching packages ---------------------------------------------------- tidyverse 1.3.0 --
\end{verbatim}

\begin{verbatim}
## v ggplot2 3.3.2     v purrr   0.3.4
## v tibble  3.0.3     v dplyr   1.0.2
## v tidyr   1.1.2     v stringr 1.4.0
## v readr   1.3.1     v forcats 0.5.0
\end{verbatim}

\begin{verbatim}
## -- Conflicts ------------------------------------------------------- tidyverse_conflicts() --
## x dplyr::filter() masks stats::filter()
## x dplyr::lag()    masks stats::lag()
\end{verbatim}

\begin{Shaded}
\begin{Highlighting}[]
\KeywordTok{library}\NormalTok{(arsenal)}
\end{Highlighting}
\end{Shaded}

\hypertarget{problem-2}{%
\subsubsection{Problem 2}\label{problem-2}}

\hypertarget{part-a}{%
\subparagraph{Part A}\label{part-a}}

\begin{Shaded}
\begin{Highlighting}[]
\CommentTok{#Import Data }
\NormalTok{Knee_df=}
\StringTok{   }\KeywordTok{read.csv}\NormalTok{(}\StringTok{"./data/Knee.csv"}\NormalTok{)}
\end{Highlighting}
\end{Shaded}

\begin{Shaded}
\begin{Highlighting}[]
\KeywordTok{summary}\NormalTok{(Knee_df)}
\end{Highlighting}
\end{Shaded}

\begin{verbatim}
##      Below       Average          Above      
##  Min.   :29   Min.   :28.00   Min.   :20.00  
##  1st Qu.:36   1st Qu.:30.25   1st Qu.:21.00  
##  Median :40   Median :32.00   Median :22.00  
##  Mean   :38   Mean   :33.00   Mean   :23.57  
##  3rd Qu.:42   3rd Qu.:35.00   3rd Qu.:24.50  
##  Max.   :43   Max.   :39.00   Max.   :32.00  
##  NA's   :2                    NA's   :3
\end{verbatim}

From the current descriptive statistics, it seems like lower physical
status are associate with longer time (days) required in physical
therapy until successful rehabilitation on average. This trend is
observable across all descriptive data (min, median,mean,max) Those who
are below average physical status have the higher value in min, median,
mean, and max for time (days) required in physical therapy when
comparing to the other group.

\hypertarget{part-b}{%
\subparagraph{Part B}\label{part-b}}

\begin{itemize}
\tightlist
\item
  Hypothesis

  \begin{itemize}
  \tightlist
  \item
    H0 = Levels of Physical status are not associated with days required
    in physical therapy until successful rehabilitation.
  \item
    H1 = Levels of Physical status are associated with days required in
    physical therapy until successful rehabilitation.
  \end{itemize}
\item
  Anova Table/Test Statistics
\end{itemize}

\begin{Shaded}
\begin{Highlighting}[]
\CommentTok{#Tidy the data frame}
\NormalTok{Knee_Anova=}
\NormalTok{Knee_df }\OperatorTok
\StringTok{  }\KeywordTok{pivot_longer}\NormalTok{(}
\NormalTok{    Below}\OperatorTok{:}\NormalTok{Above,}
    \DataTypeTok{names_to =} \StringTok{"Physical_Status"}\NormalTok{, }
    \DataTypeTok{values_to =} \StringTok{"Days_in_Physical_Therapy"}\NormalTok{) }

\CommentTok{#perform anova test}
\NormalTok{teststats =}\StringTok{ }\KeywordTok{aov}\NormalTok{(Days_in_Physical_Therapy }\OperatorTok{~}\StringTok{ }\NormalTok{Physical_Status, }\DataTypeTok{data =}\NormalTok{ Knee_Anova)}
\KeywordTok{summary}\NormalTok{(teststats)}
\end{Highlighting}
\end{Shaded}

\begin{verbatim}
##                 Df Sum Sq Mean Sq F value   Pr(>F)    
## Physical_Status  2  795.2   397.6   19.28 1.45e-05 ***
## Residuals       22  453.7    20.6                     
## ---
## Signif. codes:  0 '***' 0.001 '**' 0.01 '*' 0.05 '.' 0.1 ' ' 1
## 5 observations deleted due to missingness
\end{verbatim}

\begin{itemize}
\tightlist
\item
  Critical Value
\end{itemize}

\begin{Shaded}
\begin{Highlighting}[]
\NormalTok{F_critic=}\KeywordTok{qf}\NormalTok{(.}\DecValTok{99}\NormalTok{, }\DataTypeTok{df1=}\DecValTok{2}\NormalTok{, }\DataTypeTok{df2=}\DecValTok{22}\NormalTok{)}
\NormalTok{F_critic}
\end{Highlighting}
\end{Shaded}

\begin{verbatim}
## [1] 5.719022
\end{verbatim}

F\_critic(5.719022) \textless{} F\_stats(19.28)

\begin{itemize}
\tightlist
\item
  Decision: Since out F-stats is larger than F critical value with
  alpha=0.01, we thus reject the null and conclude that
\end{itemize}

\hypertarget{part-c}{%
\subparagraph{Part C}\label{part-c}}

Pairwise t test with bonferroni Adjustment

\begin{Shaded}
\begin{Highlighting}[]
\KeywordTok{pairwise.t.test}\NormalTok{(Knee_Anova}\OperatorTok{$}\NormalTok{Days_in_Physical_Therapy,Knee_Anova}\OperatorTok{$}\NormalTok{Physical_Status,}\DataTypeTok{p.adj =} \StringTok{"bonferroni"}\NormalTok{)}
\end{Highlighting}
\end{Shaded}

\begin{verbatim}
## 
##  Pairwise comparisons using t tests with pooled SD 
## 
## data:  Knee_Anova$Days_in_Physical_Therapy and Knee_Anova$Physical_Status 
## 
##         Above   Average
## Average 0.0011  -      
## Below   1.1e-05 0.0898 
## 
## P value adjustment method: bonferroni
\end{verbatim}

Tukey Adjustment

\begin{Shaded}
\begin{Highlighting}[]
\KeywordTok{TukeyHSD}\NormalTok{(teststats)}
\end{Highlighting}
\end{Shaded}

\begin{verbatim}
##   Tukey multiple comparisons of means
##     95% family-wise confidence level
## 
## Fit: aov(formula = Days_in_Physical_Therapy ~ Physical_Status, data = Knee_Anova)
## 
## $Physical_Status
##                    diff        lwr      upr     p adj
## Average-Above  9.428571  3.8066356 15.05051 0.0010053
## Below-Above   14.428571  8.5243579 20.33278 0.0000102
## Below-Average  5.000000 -0.4113011 10.41130 0.0736833
\end{verbatim}

Dunnett

I can't load the DescTools package, so I will calculate Dunnet by hand

Dunnett critical value(alpha=0.05,3,25) = 2.35 (From online Dunnet
critical value table) Dunnett = T\_critical\emph{sqrt(2}MS/n)

\begin{Shaded}
\begin{Highlighting}[]
\NormalTok{Dunnett =}\StringTok{ }\FloatTok{2.35} \OperatorTok{*}\StringTok{ }\KeywordTok{sqrt}\NormalTok{(}\DecValTok{2}\OperatorTok{*}\FloatTok{20.6}\OperatorTok{/}\DecValTok{3}\NormalTok{) }

\NormalTok{Dunnett}
\end{Highlighting}
\end{Shaded}

\begin{verbatim}
## [1] 8.70875
\end{verbatim}

we are setting `below average' as reference

\begin{Shaded}
\begin{Highlighting}[]
\NormalTok{average_below =}\StringTok{ }\DecValTok{33-38}
\NormalTok{Above_below =}\StringTok{ }\FloatTok{23.57}\DecValTok{-38}
\NormalTok{average_below}
\end{Highlighting}
\end{Shaded}

\begin{verbatim}
## [1] -5
\end{verbatim}

\begin{Shaded}
\begin{Highlighting}[]
\NormalTok{Above_below}
\end{Highlighting}
\end{Shaded}

\begin{verbatim}
## [1] -14.43
\end{verbatim}

Since Above and below difference exceed the magnitude of Dunnett's
distance of 8.70875, so they are the significant pairs.

\end{document}
